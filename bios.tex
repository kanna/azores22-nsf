\section{Organizing Committee and affiliations}
\label{sec:bios}

Brief backgrounds and affiliations of the organizers follow. 

\begin{longtable}{p{16.5cm}}
% \begin{longtable}{p{4.5cm} p{11cm}}  
  % \hline
  % \raisebox{-\totalheight}{\includegraphics[scale=0.045]{../../fig/aida_snow_s.png}}
\textbf{Aida Alvera-Azcárate} is a researcher at the GeoHydrodynamics
and Environment Research (GHER) group of the University of Li\`{e}ge
(Belgium). She is a physical oceanographer specialising in the
development of data analysis techniques for ocean remote sensing, like
the gap-filling technique DINEOF (DataInterpolating Empirical Orthogonal
Functions), widely used by oceanographers. She also enjoys studying the
ocean dynamics from remote sensing data and the influence of ocean
dynamics on the ecosystem. She is Associate Editor of Remote Sensing of
Environment.\newline
\textbf{email: }\emph{a.alvera@uliege.be}\newline
\textbf{Web: }\url{http://labos.ulg.ac.be/gher/aida}\\
\\
% \hline
  % \raisebox{-\totalheight}{\includegraphics[scale=0.5]{../../fig/eidsvik.jpg}}
\textbf{Jo Eidsvik} is a Professor of Statistics at NTNU, Norway. His
research profile is in spatial and computational statistics applied to
the Earth sciences. He co-authored the book on Value of Information in
the Earth Sciences (Cambridge Univ Press, 2015). He has industry
experience from the Norwegian Research Defense Establishment and from
Equinor. He is currently involved in several research projects on
spatial and spatio-temporal modeling and monitoring, including ongoing
work on autonomous maritime sampling and a center for geophysical
forecasting.\newline
\textbf{email: }\emph{jo.eidsvik@ntnu.no}\newline
\textbf{Web: }\url{https://www.ntnu.no/employees/jo.eidsvik}\\
\\
% \hline
% \raisebox{-\totalheight}{\includegraphics[scale=0.28]{../../fig/KRajan.jpg}}
\textbf{Kanna Rajan} is a Fellow at SIFT LLC, Minneapolis and holds a
visiting faculty Professorship at \unive, Portugal in autonomous
systems. At \nas his software was responsible for the command/control
of the 1999 New Millennium Deep Space 1, 65 Million miles from Earth
and the 2003 Mars Exploration Rovers mission on the Red Planet. In
2005 he moved to \mba to build the only AI group in marine robotics
and to focus on using machine intelligence for marine robotics and
biological oceanography.  His publications have been in highly ranked
peer-reviewed publications while his field work includes scientific
oceanographic cruises in the Pacific and the Atlantic. He will be a
Fulbright Fellow in Portugal in Spring 2022.\newline \textbf{email:
}\emph{Kanna.Rajan@fe.up.pt}\newline
\textbf{Web: }\url{https://kanna.rajan.systems}\\
  % \hline
\end{longtable}

