\subsection{Intellectual Merit}

Ocean science is a highly inter-disciplinary field. Yet it's practice in
the field especially in the context of \emph{observational methods}
continue to be hobbled by lack of scale or of making sense of the data
across space and time, once they have been collected. This disconnect
between the reality on the ground in the computational and robotic
sciences, with the actual needs in ocean science are stark and at the
heart of what the \symp aims to address. The principal focus of this
meeting therefore, is to address the gaps between the science needs and
the available technologies. Since ocean science is a vast field, the
core domain this symposium will focus on, is the upper water-column
bio-geochemistry.

Secondly, the typical methods of interaction while useful, can and
should be augmented by more probing participation. This can occur in
multiple ways; we intend to explore one such approach used by the
Cognitive Science community. It is also highly participatory.

Third, in this Decade of the Ocean, it provides a distinct foci for
scientists and technologists to make a coherent thrust in working
together to make concrete progress in observational methods. In large
part this is doable, because of the maturity and advances in
computational and robotic sciences, which leverage decades of funding in
the US and Europe.

\subsection{Broader Impacts}

One goal the symposium expects, is to produce a manuscript with all
participants, which exposes the ideas and concepts to the larger ocean
science community. However, a larger and more ambitious outcome we
expect and anticipate is to bring together researchers from diverse
backgrounds who might not have any reason (or incentive) to interact
outside their siloed environments. In other words, this effort is a form
of persuasion to collaborate across disciplinary and geographic
boundaries. While the latter continues to be a challenge, the
expectation is that methods in trans-national collaboration will be
provided by agencies in the US and Europe during this UN Decade. 